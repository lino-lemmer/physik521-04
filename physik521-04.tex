% Für Seitenformatierung

\documentclass[DIV=15]{scrartcl}

% Zeilenumbrüche

\parindent 0pt
\parskip 6pt

% Für deutsche Buchstaben und Synthax

\usepackage[ngerman]{babel}

% Für Auflistung mit speziellen Aufzählungszeichen

\usepackage{paralist}

% zB für \del, \dif und andere Mathebefehle

\usepackage{amsmath}
\usepackage{commath}
\usepackage{amssymb}

% für nicht kursive griechische Buchstaben

\usepackage{txfonts}

% Für \SIunit[]{} und \num in deutschem Stil

\usepackage[output-decimal-marker={,}]{siunitx}
\usepackage[utf8]{inputenc}

% Für \sfrac{}{}, also inline-frac

\usepackage{xfrac}

% Für Einbinden von pdf-Grafiken

\usepackage{graphicx}

% Umfließen von Bildern

\usepackage{floatflt}

% Für Links nach außen und innerhalb des Dokumentes

\usepackage{hyperref}

% Für weitere Farben

\usepackage{color}

% Für Streichen von z.B. $\rightarrow$

\usepackage{centernot}

% Für Befehl \cancel{}

\usepackage{cancel}

% Für Layout von Links

\hypersetup{
	citecolor=black,
	colorlinks=true,
	linkcolor=black,
	urlcolor=blue,
}

% Verschiedene Mathematik-Hilfen

\newcommand \e[1]{\cdot10^{#1}}
\newcommand\p{\partial}

\newcommand\half{\frac 12}
\newcommand\shalf{\sfrac12}

\newcommand\skp[2]{\left\langle#1,#2\right\rangle}
\newcommand\mw[1]{\left\langle#1\right\rangle}

\renewcommand \exp[1]{\mathrm{exp}\del{#1}}

% Nabla und Kombinationen von Nabla

\renewcommand\div[1]{\skp{\nabla}{#1}}
\newcommand\rot{\nabla\times}
\newcommand\grad[1]{\nabla#1}
\newcommand\laplace{\triangle}
\newcommand\dalambert{\mathop{{}\Box}\nolimits}

%Für komplexe Zahlen

\newcommand \ii{\mathrm i}
\renewcommand{\Im}{\mathop{{}\mathrm{Im}}\nolimits}
\renewcommand{\Re}{\mathop{{}\mathrm{Re}}\nolimits}

%Für Bra-Ket-Notation

\newcommand\bra[1]{\left\langle#1\right|}
\newcommand\ket[1]{\left|#1\right\rangle}
\newcommand\braket[2]{\left\langle#1\left.\vphantom{#1 #2}\right|#2\right\rangle}
\newcommand\braopket[3]{\left\langle#1\left.\vphantom{#1 #2 #3}\right|#2\left.\vphantom{#1 #2 #3}\right|#3\right\rangle}


\setcounter{section}{0}
\renewcommand\thesection{H\,4.\arabic{section}}
\renewcommand\thesubsection{\thesection.\alph{subsection}}

\title{physik521: Übungsblatt 04}
\author{%
    Lino Lemmer \\ \small{\texttt{s6lilemm@uni-bonn.de}}
    \and
    Martin Ueding \\ \small{\texttt{mu@martin-ueding.de}}
    \and
    Paul Manz \\ \small{\texttt{p.m@uni-bonn.de}}
}

\begin{document}
\maketitle

\section{Thermodynamische Relationen}

\section{Zentraler Grenzwertsatz}

\subsection{}

Mit der Wahrscheinlichkeit für das Ereignis $(x_1, \ldots, x_N)$ ist die
Wahrscheinlichkeit gemeint, mit der bei $N$ Ereignissen gerade diese in gerade
dieser Reihenfolge herauskommen? Dies ist dann:
\[
    \prod_{i=1}^N p(x_i)
\]

Bei der Wahrscheinlichkeit für ein bestimmtes $y$ kommt es auf die Reihenfolge
der $x_i$ nicht an. Daher können wir mit $N!$ multiplizieren, um alle möglichen
Permutationen zusammenzufassen.

Je nach der Struktur der Menge $M$ könnte es auch sein, dass verschiedene
Ergebnismengen $\set{x_i \colon i = [1, N] \cap \mathbb N}$ zum gleichen $y$
führen, aber das kann man nicht genau sagen.

Also erhalten wir:
\[
    P_N(y) = \frac{N!}{N} \prod_{i=1}^N p(x_i)
\]

\subsection{Charakteristische Funktion}

Wir nehmen die Definition der charakteristischen Funktion und setzen unser
$P_N$ dort ein:
\begin{align*}
    X_N(k)
    &= \int \exp{\ii k \frac 1N \sum_{i=1}^N (x_i - \langle x \rangle)}
    \frac{N!}N \prod_{i=1}^N p(x_i) \dif x_i \\
    \intertext{%
        Die Summe im Exponenten kann man in ein Produkt von
        Exponentialfunktionen schreiben.
    }
    &= \frac{N!}N \prod_{i=1}^N \underbrace{\int 
        \exp{\ii k \frac 1N (x_i - \langle x \rangle)}
    p(x_i) \dif x_i}_{\chi\del{\frac kN}} \\
    \intertext{%
        Produkte gleicher Faktoren sind Potenzen.
    }
    &= \frac{N!}N \del{\chi\del{\frac kN}}^N
\end{align*}

Dies stimmt bis auf unseren Vorfaktor auch mit dem Ergebnis überein, das wir
erhalten sollten.

\end{document}

% vim: spell spelllang=de tw=79
